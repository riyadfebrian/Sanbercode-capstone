\documentclass{report}
\usepackage[utf8]{inputenc}
\usepackage{indentfirst}

\title{Tugas 1 Python \& Data Science Bootcamp}
\author{Riyad Febrian - riyadfebrian@gmail.com}
\date{July 2020}

\begin{document}

\maketitle


\begin{enumerate}
    \item Apa yang kamu ketahui tentang Web Scraping ?
    
        \emph{Webscraping} merupakan salah satu teknik dalam \emph{data mining} atau penggalian data dengan melakukan ekstraksi konten dari html web
    
    \item Sebutkan \emph{library} apa saja yang bisa digunakan untuk melakukan \emph{web scraping}
    
        \emph{library} yang paling populer untuk web scraping adalah \textbf{beautifoulSoup}, sementara jika yang ingin kita \emph{scrape} hanya tag \emph{table} dari html web, maka \textbf{pandas} lebih mudah untuk digunakan.
    
    \item Sebutkan semua tag html yang kamu ketahui beserta kegunaannya
    
        Tag html yang biasanya digunakan untuk keperluan scraping :
        \begin{itemize}
            \item \emph{\textless body\textgreater} :
            tag ini mengandung konten utama keseluruhan dari sebuah html
            
            \item \emph{\textless div\textgreater} :
            tag ini mengandung konten-konten yang lebih spesifik, biasanya dibarengi dengan \emph{property class} atau \emph{id}. Hal ini, memudahkan untuk menyeleksi tag tersebut berdasarkan \emph{class} atau \emph{id}-nya
            
            \item \emph{\textless h1\textgreater} \ldots \emph{\textless h6\textgreater} :
            tag ini untuk \emph{heading}, biasanya dipakai untuk penjudulan. semakin besar nomor headingnya, maka ukuran \emph{fontsize}-nya semakin kecil atau \emph{subheading} dari nomor sebelumnya.
            
            \item \emph{\textless p\textgreater} :
            tag ini mengandung konten paragraf teks
            
            \item \emph{\textless table\textgreater}  :
            tag ini mengandung konten berupa data tabular, tag didalamnya termasuk \emph{table head} \textless th\textgreater, \emph{table row} \textless tr\textgreater, dan \emph{table data} \textless td\textgreater
            
              \item \emph{\textless ul\textgreater dan \textless ol\textgreater } :
            tag \textless ul\textgreater dipakai untuk menampilkan poin-poin dalam \emph{bullet point} sementara \textless ol\textgreater menampilkan poin-poin dalam format \emph{numbering} \emph{i.e} 1..2..dst
            
            \item \emph{\textless a\textgreater} :
            tag ini mengandung referensi link
    \end{itemize}
    
    \item Sebutkan pengalaman kamu dalam bahasa pemrograman python
    
        Pengalaman pertama kali menggunakan python tahun 2015, dimana saat itu materi python merupakan bagian dari mata kuliah administrasi jaringan, jadi implementasi python-nya lebih ke \emph{networking} seperti membuat \emph{basic} TCP, UDP, dll.
        
        Sejujurnya, di jurusan saya diajarkan \emph{multilingual language} dari bahasa paling low level seperti Assembly sampai ke high level seperti C++, Java, Python. Suatu saat saya berfikir bahwa saya nggak bisa menjadi orang yang \emph{jack of all trades}, saya harus memilih 1 bahasa dan menguasai bahasa tersebut. Di tahun 2018, saya mulai tertarik untuk mendalami python dan menjatuhkan pilihan untuk menguasai python setelah terekspos soal AI/Machine Learning dan Data Science. 
        
        beberapa \emph{Project} yang telah saya kerjakan yaitu \emph{Scraping} Data gejala penyakit dari web HelloSehat, Implementasi Reinforcement Learning pada Quadcopter, analisis beberapa project \emph{basic data science}.
    
    \item Kelas python lanjutan seperti apa yang kamu harapkan di SanberCode kedepannya ?
    
        Disamping mengenalkan peserta dengan teknologi Deep learning dan lainnya, Harapannya bisa lebih membahas implementasi teknik-teknik statistika tingkat lanjut untuk handle data tertentu atau \emph{improve} machine learning dengan studi kasus dari \emph{real project}. 
\end{enumerate}

\end{document}
