\documentclass{report}

\usepackage{enumitem}

% package for code
\usepackage{listings}
\usepackage[most]{tcolorbox}

    
% json listing
\colorlet{punct}{red!60!black}
\definecolor{background}{HTML}{EEEEEE}
\definecolor{delim}{RGB}{20,105,176}
\colorlet{numb}{magenta!60!black}

\lstdefinelanguage{json}{
    basicstyle=\normalfont\ttfamily,
    numbers=left,
    numberstyle=\scriptsize,
    stepnumber=1,
    numbersep=8pt,
    showstringspaces=false,
    breaklines=true,
    frame=lines,
    backgroundcolor=\color{background},
    literate=
     *{0}{{{\color{numb}0}}}{1}
      {1}{{{\color{numb}1}}}{1}
      {2}{{{\color{numb}2}}}{1}
      {3}{{{\color{numb}3}}}{1}
      {4}{{{\color{numb}4}}}{1}
      {5}{{{\color{numb}5}}}{1}
      {6}{{{\color{numb}6}}}{1}
      {7}{{{\color{numb}7}}}{1}
      {8}{{{\color{numb}8}}}{1}
      {9}{{{\color{numb}9}}}{1}
      {:}{{{\color{punct}{:}}}}{1}
      {,}{{{\color{punct}{,}}}}{1}
      {\{}{{{\color{delim}{\{}}}}{1}
      {\}}{{{\color{delim}{\}}}}}{1}
      {[}{{{\color{delim}{[}}}}{1}
      {]}{{{\color{delim}{]}}}}{1},
}

\title{Tugas 9 Python \& Data Science Bootcamp}
\author{Riyad Febrian - riyadfebrian@gmail.com}
\date{July 2020}

\begin{document}

\maketitle

\begin{enumerate} [label=\Alph*]
    \item Soal
    \begin{enumerate} [label=\arabic*]
        \item Ceritakan kembali apa yang kalian ketahui tentang JSON
        \item Data Jouska yang telah kalian ambil pada tugas ke-8 coba kalian ubah ke dalam dataframe pandas lalu diubah kedalam JSON. Dengan ketentuan:
            \begin{itemize}
                \item Jumlah tweetnya 50
                \item Diambil dari tanggal 25 Juli
                \item Dalam dataframe harus ada kolom User, Lokasi, dan Tweet
                \item Ubah Data tersebut kedalam JSON
                \item Hasil print(JSON) itulah yang diisikan kedalam file PDF
            \end{itemize}
    \end{enumerate}
    
    \item Jawaban
        \begin{enumerate}[label=\arabic*]
            \item  \textbf{JSON} merupakan \emph{Javascript Object Notation} yang digunakan sebagai format data dalam melakukan pertukaran dan penyimpanan data. JSON bisa diibaratkan sebagai bahasa bersama (\emph{global}) yang dimengerti oleh semua sistem, sehingga memudahkan interaksi data ketika menggunakan format JSON. Selain itu, JSON juga lebih \emph{human readable}
            
            \item Hasil JSON
              \lstinputlisting[language=json, firstnumber=1]{result_JOUSKA.json}
        \end{enumerate}
        
        
    

\end{enumerate}


\end{document}
